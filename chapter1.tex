\chapter{Dynamic cloud-based scaling web-service}

\textbf{Keywords: Virtualization, Cloud computing, performance, scripting} 

\subsection*{Abstract}
This report takes a look at the implementation of a dynamic setup in cloud-based
web-services that scale with the load. With the loadbalancer HAproxy and
implemented through OpenStack APIs.

\section{Introduction}

\subsubsection{Problem statement}

The given problem statement in this project was as follows:

\emph{Build a cloud-based web service which is able to adjust the number of
webservers based on the incoming rate of user requests.}

\section{Background}
When having a large site with many visitors or resource demanding calculations
there is a need to be able to run the calculations on multiple servers. Running
just a single machine will result in slow response time, and possible loss of
service. To solve this, multiple servers serving sites needs to be used. This
is done through the use of load balancers. 



\subsection{Benchmarking webservers}

\subsection{httperf}

\subsection{HAproxy}

\subsection{Cloud}

\section{Approach}

\section{Result}


\section{Analysis}

\section{Discussion and conclusion}

\subsection{Conclusion}
