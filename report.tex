\documentclass[11pt,english,a4paper] {report}
\usepackage[latin1]{inputenc}
\usepackage[T1]{fontenc}
\usepackage{babel,graphicx,varioref}
\usepackage{palatino}
\usepackage{relsize}
\usepackage{varioref}
\usepackage{pdfpages}
\usepackage{fancyvrb}
\usepackage{fancyhdr}
\usepackage{sectsty}
\usepackage{times}
\usepackage{hyperref}
\usepackage{amsmath}
\usepackage{geometry}

%For code listings
\usepackage{color}
\usepackage{xcolor}
\usepackage{listings}
\usepackage{courier}
\usepackage{pdfpages}
\usepackage{varioref}

\usepackage{dirtree}

\definecolor{codegreen}{rgb}{0,0.6,0}
\definecolor{codegray}{rgb}{0.5,0.5,0.5}
\definecolor{codepurple}{rgb}{0.58,0,0.82}
\definecolor{backcolour}{rgb}{0.95,0.95,0.92}
\definecolor{keywordcolor}{RGB}{133,0,3}
\definecolor{commentcolor}{RGB}{135,136,117}
\definecolor{numbercolor}{RGB}{38,38,38}
\definecolor{stringcolor}{RGB}{211,0,53}

%\renewcommand*\rmdefault{Consolas, 'Liberation Mono', Menlo, Courier, monospace}

\def\ff{Consolas, 'Liberation Mono', Menlo, Courier, monospace}

\lstset{
    basicstyle=\scriptsize,%\fontfamily{\ff},
    numbers=left,               % Ort der Zeilennummern
    numberstyle=\scriptsize,%\tiny,          % Stil der Zeilennummern
    %stepnumber=2,               % Abstand zwischen den Zeilennummern
    numbersep=5pt,              % Abstand der Nummern zum Text
    tabsize=2,                  % Groesse von Tabs
    extendedchars=true,         %
    breaklines=true,            % Zeilen werden Umgebrochen
    %keywordstyle=\color{red},
    frame=b,         
    keywordstyle=[1]\textbf,    % Stil der Keywords
    keywordstyle=[2]\textbf,    %
    keywordstyle=[3]\textbf,    %
    %keywordstyle=[4]\textbf,   \sqrt{\sqrt{}} %
    %stringstyle=\color{white}\ttfamily, % Farbe der String
    showspaces=false,           % Leerzeichen anzeigen ?
    showtabs=false,             % Tabs anzeigen ?
    xleftmargin=17pt,
    framexleftmargin=17pt,
    framexrightmargin=5pt,
    framexbottommargin=4pt,
    %backgroundcolor=\color{lightgray},
    showstringspaces=false,%false      % Leerzeichen in Strings anzeigen ?        
    breakatwhitespace=false,
    keepspaces=true,
    commentstyle=\color{commentcolor}\fontfamily{\ff},%codegreen},
    keywordstyle=\color{keywordcolor}\fontfamily{\ff}, %magenta},
    numberstyle=\tiny\color{numbercolor}\fontfamily{\ff}, %codegray
    stringstyle=\color{stringcolor}\fontfamily{\ff}%codepurple
 }
 \lstloadlanguages{
        Python,
        Bash
 }
    %\DeclareCaptionFont{blue}{\color{blue}} 

  %\captionsetup[lstlisting]{singlelinecheck=false, labelfont={blue}, textfont={blue}}
  %\usepackage{caption}
%\DeclareCaptionFont{white}{\color{white}}
%\DeclareCaptionFormat{listing}{\colorbox[cmyk]{0.43, 0.35, 0.35,0.01}{\parbox{\textwidth}{\hspace{15pt}#1#2#3}}}
%\captionsetup[lstlisting]{format=listing,labelfont=white,textfont=white, singlelinecheck=false, margin=0pt, font={bf,footnotesize}}
\usepackage{caption}
\DeclareCaptionFont{white}{\color{white}}
\DeclareCaptionFormat{listing}{\colorbox{gray}{\parbox{\textwidth}{#1#2#3}}}
\captionsetup[lstlisting]{format=listing,labelfont=white,textfont=white}
% done code

%\usepackage[colorlinks=true,pdfstartview=FitH, linkcolor=blue, 
            %citecolor=blue, urlcolor=blue,bookmarksopen=true]{hyperref}
\def\course{MS016A}
\def\reportname{Lab Report}
\def\name{Lars Haugan}
\author{Lars Haugan - s171201}
\def\studentnr{s171201}
\def\line{NSA}
\def\school{University of Oslo \\ Oslo and Akershus Univeristy College of Applied Sciences}
\def\semester{Autumn 2014}

\hypersetup
{
        pdftitle={\reportname},
        pdfauthor={\name},
        pdfsubject={\reportname},
        colorlinks=true,
        linkcolor=blue,
        citecolor=black,
        urlcolor=blue,
        pdfstartview=FitH,
        bookmarksopen=true
}

\usepackage{apacite}
\bibliographystyle{apacite}

\parindent=0in
\fvset{frame=single,framesep=3mm,fontfamily=helvetica,fontsize=\scriptsize}

\renewcommand\ttfamily{\small\bf\fontfamily{helvetica}}
\newcommand{\HRule}{\rule{\linewidth}{0.2mm}}

\begin{document}
\begin{titlepage}
    \begin{center}

        \large \line\\
        \large \school\\ \semester\\[0.4cm]
        \HRule\\[1.5cm]
        \name, \studentnr
        \ \\[6.0cm]
        \LARGE\textbf{\course\ - \reportname}

    \end{center}
\end{titlepage}


%\title{  \course\ \reportname } %\\ \name \ - \studentnr }

\allsectionsfont{\sffamily}

\tableofcontents

\chapter{Dynamic cloud-based scaling web-service}

\textbf{Keywords: Virtualization, Cloud computing, performance, scripting} 

\subsection*{Abstract}


\section{Introduction}

\subsubsection{Problem statement}

The given problem statement in this project was as follows:

\emph{Build a cloud-based web service which is able to adjust the number of
webservers based on the incoming rate of user requests.}

\section{Background}

\subsection{Benchmarking webservers}

\subsection{httperf}


\section*{Approach}

\section{Result}


\section{Analysis}

\section{Discussion and conclusion}

\subsection{Conclusion}

\chapter{Comparing HAProxy and Pound load balancers}
\textbf{Keywords: Webservers, Performance, Analysis}

\subsection*{Abstract}

\section{Introduction}

\subsubsection{Problem statement}

The given problem statement in this project was as follows:

\emph{Setup and evaluate and compare the HAproxy and pound load balancers for a
web- service with regard to:
\begin{itemize}
    \item Configureability
    \item Performance
    \item Scalability
\end{itemize}}

\section{Background}
Load balancers are important to most of the high end websites that are
available. Ensuring high availability and scalability. They can be a part
of even the smallest sites that are prone to fail, ensuring high uptime,
or to balance the load over multiple webservers, or even database servers.

There are many different load balancers available. Many of these are open source
and free, but with varying quality and support. There are also many enterprise
load balancers or reverse proxy´s as they are usually referenced to, like
Alteon product from Radware and Big IP from F5. But a load balancer could be as
simple as a single web server running apache or nginx.

% why do we need to compare the balancers.

% Talk about httperf
\subsection{Performance testing with httperf}

% Talk about HAProxy
\subsection{HAProxy}

% Talk about Pound
\subsection{Pound}

\section{Approach}
In this comparison the two load balancers HAproxy and Pound will be compared on
the different metrics of c
% How thing is supposed to be set up
% Hypothesis

% assumption: set of assumption for the different cases
% how the setup is
% The different cases:
    % httperf with increasing rate of requests and/or connections
        % naturally that there are multiple requests per connection

\section{Result}
%What happened
% os limits
% sudo apt-get install autoconf libtool -y
% https://rtcamp.com/tutorials/benchmark/httperf/
% before make install run ulimit -n 65535

% Subsection: Configurability
%   Syntax? Functions, documentation, community and continuous support (new
%   versions?) This is relevant, but not directly to how it works. Evaluate
%   this!
%
\subsection{Configurability}

\subsection{Performance}

\subsection{Scalability}

% the data are presented as described in the approach section
% how are the tests run
% sample output of scripts and/or httperf
% present the data for the spesifik sets
    % how to find the data and how they were analyzed


% ssl signing: openssl req -x509 -sha256 -nodes -days 365 -newkey rsa:4096
% -keyout mycert.pem -out mycert.pem

\section{Analysis}
 %Look at the data
 % Analyze the hypothesis

\section{Discussion and conclusion}


\subsection{Improvements}
% SSL and why we dont care here. 
% Other possible benchmarking tools?

\subsection{Conclusion}

\section{Appendix}


\bibliography{references}

\end{document}
