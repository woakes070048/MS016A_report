\chapter{A monitoring tool for /proc/diskstats}
\textbf{Keywords: Storage, Performance}

\subsection*{Abstract}
In this report, a tool for monitoring the \textit{/proc/diskstats} file is
developed, storing the content for later analysis.

\section{Introduction}

\subsubsection{Problem statement}

The given problem statement in this project was as follows:

\emph{Write a script which is able to collect data from /proc/diskstats and store 
it in an orderly manner for later analysis.}

\section{Background}
% describe the task at hand
While running a system it is sometimes important to be able to gather metrics
on how the system is performing. One of this metrics are the data from disk
operations. This can help identify bottlenecks in the system, and give a better
understanding of how the system is using its resources.

To understand the usage and need for a tool that can monitor the file
\textit{/proc/diskstat} a short introduction to the files under \textit{/proc}
and the values stored inside is needed.

\subsection{The /proc folder}
/proc is a virtual filesystem, which sometimes is reffered to as a process
information pseudo-file system \cite{tldp:proc}. The files in the folder are
not \textit{real}, but is rather a means of getting information from the
kernel.

\subsection{/proc/diskstats}
The file /proc/diskstats contain information about each block device on the
system. In newer kernels (from 2.6 and newer) this information is also available
under the path /sys/block/. Each line has the different information about each
block device, which could either be the disk itself, or the partitions on it.
The information for each block device displays the I/O statistics like reads,
writes and time spent doing the different tasks.

The different fields of the file means the following:
\begin{enumerate}\itemsep1pt \parskip0pt \parsep0pt
    \item major number
    \item minor mumber
    \item device name
    \item reads completed successfully
    \item reads merged
    \item sectors read
    \item time spent reading (ms)
    \item writes completed
    \item writes merged
    \item sectors written
    \item time spent writing (ms)
    \item I/Os currently in progress
    \item time spent doing I/Os (ms)
    \item weighted time spent doing I/Os (ms)
\end{enumerate}
\label{lbl:column_description}

These are then representing each column in the file which can look like the
following:
%\begin{Verbatim}[label=Contents of /proc/diskstats]
\begin{center}
\begin{lstlisting}[label=lbl:content, caption=Contents of /proc/diskstats,
numbers=none]
 253       0 vda 31410 806 1020922 2564596768 1465561 1382324 54703840 3579730636 0 7229708 40955272
 253       1 vda1 31182 334 1015322 2564547860 1465459 1382324 54703024 3579730152 0 7229220 40954784
\end{lstlisting}
\end{center}
%\end{Verbatim}
\label{lbl:content}

There are other tools which enables the logging of different metrics from block
devices as well, like the \textit{sar} tool and \textit{iostat}. One of the
benefits the monitoring of /proc/diskstats holds over the usage of other tools,
is the low overhead, and no need for additional packages.

This tool will enable the continuous logging of the kernel counters for the
block devices specified, and prepare the data for later analysis.

\section{Approach}
% How thing is supposed to be set up
% Hypothesis
% How is the data supposed to look like and how to analyze the data
% How is the chosen approach provide an answer to the problem statement

The operationalization asks for a tool, which can read from the /proc/diskstats
file and store the data appropriately for later analysis. The tool should be
developed in Python, as this is one of the common scripting languages used. It
also enables easy file operations functionality which is important to this
task, as the information gathered will be in the format of a file.

The intention should be that the tool could be used in two ways. The normal
usage, which would be user execution of the tool. The other way is the
possibility of running the tool as a cron job. If the tool is run with a
specific interval executed by the user, this would result in a process staying
alive in the background the hole time. Running the tool with cron, would on the
other hand only require the tool to be running at the time of execution, while
the user would need to have a session open over time if executed by the user.
This is easily done with applications like \textit{screen} or \textit{tmux},
but the usage of cron is a better way of doing it if the application should be
running over longer periods of time.

\subsection{The format of data}
As shown in the background \vref{lbl:content} the content of
/proc/diskstats is containd in lines and columns. Since each line represents a
single block device, the columns of this line contains the information about
the device. This enables the usage of iterations in python, where each column
can be splitted on the separators, which is multiple spaces.

The data can therefor be represented as a dictionary in Python. This is a
key:value based data structure. A single block device can therefor have a
single dictionary with each value of the /proc/diskstats file beeing
represented as column:data. Each column get its name from the documentation
for /proc/diskstats shown on \vref{lbl:column_description}.

\subsection{Storing the data}
To be able to analyze the results at a later point, the output of the tool
should be stored to file. One of the criteria for storing the data to file
would be the possibility of using other tools to analyze the data at a later
time. Wether the tool is written in Python, Perl or any other language should
not matter. The most common possibilities for storing to file would be
\textit{json}, \textit{xml} or \textit{csv}. Both \textit{json} and
\textit{xml} uses a key:value representation, but also makes it harder to
format the data without the usage of external libraries. There are also
numerous ways of interpreting these different formats, as there are multiple
standards. The choice is therefor to use \textit{csv} which is a simple text
format which uses the comma notation to separate each value. The top of the
file should contain the description of each column, and each field should be
represented with either a comma or 0 if there is no data.

If we takes the previous examples given in the background \vref{lbl:content},
the final file will look something like the following:
%\begin{Verbatim}[label=Possible output of monitoring tool]
\begin{center}
\begin{lstlisting}[label=lbl:possible_output,caption=Possible output of
monitoring tool,numbers=none]
date, major number,minor mumber,device name,reads completed successfully,reads merged,sectors read ...
date, 253,0,vda,31410,806,1020922,2564596768,1465561,1382324,54703840,3579730636,0,7229708,40955272
date, 253,1,vda1,31182,334,1015322,2564547860,1465459,1382324,54703024,3579730152,0,7229220,40954784
\end{lstlisting}
\end{center}
%\end{Verbatim}

Including the output from /proc/diskstats, the time stamp of when the test has
been run is important to include, as this will provide the time interval
between the two following tests to be calculated.
The collected data could with this format be imported into any scripting
language or applications like libreOffice for further work.

\subsection{Different command line options}\label{lbl:mon_command_line_opts}
As described in the beginning of this section, the tool should be able to be
run in two different modes, the user-mode and cron-mode. The tool does not need
to be implemented any specific way for it to work in the different modes, other
than actually taking some different command line options.
The needed options is to specify if the application should run the gatherings
in a loop or not. This is needed only in user-mode and can then take the
parameter as for how long the tool should sleep between each data gathering.
This is not need in cron-mode as this is specified in the cron job itself.

The other needed option is to specify which devices that the tool should gather
information on. By default the tool should store information about all the
devices, but as a command line option it should also be possible to specify a
list of devices to focus on. 

\section{Result}
%What happened
% what was done
% partial output of the commands
This section will present the actual tool that has been developed. It has been
developed in Python, and does not require any additional modules to work. This
enables it to work at any distribution, with basic Python installed.

Based on the findings in the approach, the tool has been implemented with the
following command line options, which were described in
\vref{lbl:mon_command_line_opts}. This different options is visible with the
use of \textit{-h} or \textit{--help} when running the tool.
%\begin{Verbatim}[label=Command line options of diskstats.py]
\begin{center}
\begin{lstlisting}[label=lbl:diskstats_help,caption=Command line options of
disksts.py,columns=fullflexible,numbers=none]
$ python diskstats.py -h
usage: diskstats.py [-h] [-l sleep] [-d name [name ...]]

Parser tool for /proc/diskstats

optional arguments:
  -h, --help            show this help message and exit
  -l sleep, --loop sleep
                        Runs the program in a loop. Takes the looptime as
                        option.
  -d name [name ...], --device name [name ...]
                        Names, like sda, sda1...
\end{lstlisting}
\end{center}
%\end{Verbatim}

The tool can therefore be run with the intention of running every second and
only look at the information for the block device vda. The command for this
would be the following.

%\begin{Verbatim}[label=Execution of diskstats.py]
\begin{center}
\begin{lstlisting}[label=lst:,caption=Execution of disksts.py,numbers=none]
$ python diskstats.py -l 1 -d vda
Gathering data for ['vda']
Gathering data for ['vda']
\end{lstlisting}
\end{center}
%\end{Verbatim}

When running with this options, a file with the name \textit{mon\_vda.csv} would
be created. The contents of the file would be in csv format, and formated as
described in the approach.

%\begin{Verbatim}[label=Gathered data stored in file from diskstats.py]
\begin{center}
\begin{lstlisting}[label=lst:out_data,caption=Gathered data stored in file from
diskstst.py,numbers=none]
datetime,major_number,minor_number,device_name,read_completed_successfully,reads_merged,sectors_read,time_spent_reading(ms),writes_completed,writes_merged,sectors_written,time_spent_writing_(ms),IO_currently_in_progress,time_spent_doing_IO_(ms),weighted_time_spent_doing_IO
2014-12-14T14:27:07.878920,253,0,vda,5964436,13925,540485938,2768828,1289917,1519024,48384848,9417637,0,1382484,12180809
2014-12-14T14:27:08.881174,253,0,vda,5964436,13925,540485938,2768828,1289934,1519049,48385184,9417668,0,1382487,12180840
\end{lstlisting}
\end{center}
%\end{Verbatim}

As described it should also be possible to run the script as a cron job.
\begin{center}
\begin{lstlisting}[label=lst:cronjob,caption=Cronjob example, numbers=none]
# Diskstats cronjob
# This cronjob would execute every minute, and capture only the information of vda
*/1 * * * * /usr/bin/python /path/to/diskstats.py -d vda
\end{lstlisting}
\end{center}

The final version of the tool is attached in the appendix
\vref{lst:diskstats.py}. Most of the logic that has been implemented is related
to the different command line options, as the tool needs different 

\section{Analysis}
 %Look at the data
 % Analyze the hypothesis
The aim of this tool were to be able to collect the information that was
possible to be gathered from the /proc/diskstats file. It is able to collect
the data, and store it as a csv file, which is a readable format that is
supported by many different softwares.
The script does also implement many different function beyond what is outlined
in the initial description of the needed tool, as the possibility for it being
run as a cron job, and with variable sleeping lengths.

However there are one issue that has been discovered. When running the tool as
a cron job, there is no way to specify the path to where the file should be
stored. The output of the tool is therefor lost.

This has been fixed, by adding another command line option,
\textit{--outfilepath} or \textit{-o}.

\section{Discussion and conclusion}
This tool is intended to enable the user to store the contents of the file
/proc/diskstats. This tool enables the gathering of the information for later
analysis, by storing all the specified content of the file.
By default the application runs in a simple manner, but also includes aditional
functionality, making it possible to use the tool in many different situations
where debugging of file system is important.

This tool does the same thing as many other tools available, but at a lower
level. Most of the tools available, will do some calculations on the data, but
this tool only gathers the data for later analysis.

This is in many cases a benefit. This will also make less of an impact on the
general performance of the system. This said, the performance impact of other
tools have not been tested here.

%\subsection{Improvements}
%The improvements to the tool would be outside the scope of this project, as the
%next thing would be the implementation of differentiation between the different
%collected metrics. The data on its own does not mean much 

\subsection{Conclusion}
In summary, the task has been completed to the specified requirements by
solving the following main points.
\begin{itemize}
    \item Collecting the information from \textit{/proc/diskstats}
    \item Storing the data in a orderly manner for later analysis
\end{itemize}

This conclude a simple tool that can capture the data from the kernel
file system for later analysis. The tool work as intended, and can capture the
content of the file over longer periods of time.

\section{Appendix}
%\newgeometry{margin=2cm}
\thispagestyle{empty}
\begin{center}
\lstinputlisting[label=lst:diskstats.py,caption=diskstats.py,
language=Python]{chapter3/diskstats/diskstats.py}
\end{center}
%\restoregeometry
